\documentclass{beamer} % Rename the file to remove spaces and special characters, e.g., 'encontro1-somatorio-definicao-e-propriedades.tex'
\usepackage{amsmath, graphicx}
\usetheme{Madrid}

\title{Somatórios}
\author{Professores Claudia Tavares e Luiz Otávio Rodrigues Alves Sereno}
\date{Curso: Ciência da Computação}

\begin{document}

\begin{frame}
    \titlepage
    \vfill
    \begin{center}
    \vspace{-0.9cm}
        \includegraphics[width=0.25\textwidth]{logopuc.png} \hspace{1.4cm}
        \includegraphics[width=0.25\textwidth]{logoicei.jpg}
    \end{center}
\end{frame}

\begin{frame}{Definição}
    
    O \textbf{somatório} é uma notação matemática usada para representar a soma de uma sequência de termos. Ele é indicado pelo símbolo \(\displaystyle \sum\) (sigma maiúsculo) e é definido da seguinte forma:
    $$
    \displaystyle \sum_{i=a}^{b} f(i)
    $$
    Isso significa que estamos somando os valores da função \( f(i) \) para cada \( i \) inteiro no intervalo de \( a \) até \( b \).
\end{frame}

\begin{frame}{Exemplo 1: Somatório de uma sequência de números}
    
Se quisermos somar os primeiros 5 números naturais, podemos escrever:

    \[
    \displaystyle \sum_{i=1}^{5} i = 1 + 2 + 3 + 4 + 5 = 15
    \]
\end{frame}

\begin{frame}{Exemplo 2: Somatório de uma função}
    
    
    Se tivermos a função \( f(i) = i^2 \), o somatório de \( i^2 \) de 1 a 3 seria:
    \[
    \displaystyle \sum_{i=1}^{3} i^2 = 1^2 + 2^2 + 3^2 = 1 + 4 + 9 = 14
    \]
\end{frame}

\begin{frame}{Atividades}
    
\textbf{Expanda e calcule o valor das seguintes somas:}
    \begin{itemize}
        \item $ \displaystyle \sum_{k=1}^{5} k $
        \item $ \displaystyle \sum_{n=1}^{4} (2n + 1)$
        \item $ \displaystyle \sum_{i=1}^{6} 3 $
    \end{itemize}
\end{frame}
    
\begin{frame}{Atividades}
    
\textbf{Reescreva as somas a seguir na notação de somatório:}
    \begin{itemize}
        \item $ 2 + 4 + 6 + 8 + 10 $
        \item $ 1^2 + 2^2 + 3^2 + \dots + 10^2 $
        \item $ 3 + 6 + 9 + 12 + 15 + 18 $
    \end{itemize}
\end{frame}

\begin{frame}{Propriedades}
    
As propriedades do somatório nos ajudam a manipulá-lo e a encontrar o valor da soma.

\end{frame}

\begin{frame}{Propriedade 1. Somatório de uma Constante}
	
	Se somarmos uma constante \( c \) \( n \) vezes, o resultado será:
	
	\[
	\sum_{k=1}^{n} c = c \cdot n
	\]
	\textbf{Exemplo}  
	\[
	\sum_{k=1}^{7} 5 = 5 \times 7 = 35
	\]
\end{frame}

\begin{frame}{Propriedade 2. Somatório do Produto por uma Constante}
	
	Se \( k \) for uma constante e \( a_i \) uma sequência qualquer, então:
	
	\[
	\sum_{i=p}^{n} k \cdot a_i = k \cdot a_p + k \cdot a_{p+1} + k \cdot a_{p+2} + \dots + k \cdot a_n
	\]
	
	Reescrevendo:
	
	\[
	\sum_{i=p}^{n} k \cdot a_i = k \cdot \left( a_p + a_{p+1} + a_{p+2} + \dots + a_n \right)
	\]
	
	Ou seja:
	
	\[
	\sum_{i=p}^{n} k \cdot a_i = k \cdot \sum_{i=p}^{n} a_i
	\]

\end{frame}    
	
\begin{frame}{Propriedade 2: Somatório do Produto por uma constante}
	\textbf{Exemplo}  
    
	\[
	\sum_{i=1}^{5} 3 \cdot i = 3 \sum_{i=1}^{5} i = 3(1+2+3+4+5)=45
	\]

    \end{frame}
	
\begin{frame}{Propriedade 3. Somatório de uma Soma Algébrica}
	
	Se \( a_i \) e \( b_i \) são duas sequências, então:
	
	\[
	\sum_{i=p}^{n} (a_i \pm b_i) = (a_p \pm b_p) + (a_{p+1} \pm b_{p+1}) + (a_{p+2} \pm b_{p+2}) + \dots + (a_n \pm b_n)
	\]
	Reescrevendo	
	\[
	\sum_{i=p}^{n} (a_i \pm b_i) = \sum_{i=p}^{n} a_i \pm \sum_{i=p}^{n} b_i
	\]

\end{frame}

\begin{frame}{Propriedade 3. Somatório de uma Soma Algébrica}
	\textbf{Exemplo}  
	\[
	\sum_{i=1}^{4} (2i + 3) = \sum_{i=1}^{4} 2i + \sum_{i=1}^{4} 3
	\]
	
	Calculando separadamente:
	
	\[
	\sum_{i=1}^{4} 2i = 2(1+2+3+4) = 2(10) = 20
	\]
	
	\[
	\sum_{i=1}^{4} 3 = 3.4 = 12
	\]
	
	Logo:
	
	\[
	\sum_{i=1}^{4} (2i + 3) = 20 + 12 = 32
	\]
\end{frame}	
\begin{frame}{Propriedade 4: Separação do Último Termo}
	
	A soma de termos de \( p \) até \( n \) pode ser escrita como:
	
	\[
	\sum_{i=p}^{n} a_i = \sum_{i=p}^{n-1} a_i + a_n
	\]
	
	\textbf{Exemplo:}
	
	$$
	\sum_{i=1}^{5} i = \sum_{i=1}^{4} i + 5 = 10 + 5 = 15
        $$
\end{frame}
	
\begin{frame}{Propriedade 4: Separação do Primeiro Termo}
	
	A soma de termos de \( p \) até \( n \) pode ser escrita como:
	
	$$
	\sum_{i=p}^{n} a_i = a_p + \sum_{i=p+1}^{n} a_i
	$$
	
	\textbf{Exemplo:}
	
	$$
	\sum_{i=1}^{5} i = 1 + \sum_{i=2}^{5} i = 1 + 14 = 15
	$$

\end{frame}
	
%\begin{frame}{Propriedade Telescópica}
	
%	A propriedade telescópica ocorre quando somamos termos da forma 
%    $$ 
%    a_{k+1} - a_k
%    $$
%	Temos que
%	$$
%	\sum_{i=p}^{n} (a_{i+1} - a_i) =
%    $$

%    $$
%	=(a_{p+1} - a_p) + (a_{p+2} - a_{p+1}) + (a_{p+3} - a_{p+2}) + \dots + (a_{n} - a_{n-1}) +(a_{n+1} - a_n)
%	$$
	
%	Observamos que todos os termos intermediários se cancelam, restando:
	
%	\[
%	=a_{n+1} - a_p
%	\]
	
%	Ou seja:
	
%	\[
%	\sum_{i=p}^{n} (a_{i+1} - a_i) = a_{n+1} - a_p
%	\]
%    \end{frame}

    
\begin{frame}{Atividades}
	\begin{enumerate}
		\item Calcule os seguintes somatórios
		\begin{enumerate}
			\item $\displaystyle \sum_{i=1}^{20} 1$\\
			\item $\displaystyle \sum_{i=5}^{20} 1$ \\
			\item $\displaystyle \sum_{i=m}^{m+1} 1$\\

		\end{enumerate}
	
		\item Utilize a decomposição de frações para calcular:
		\[
		\sum_{k=1}^{10} \frac{1}{k(k+1)}
		\]

	\end{enumerate}

\end{frame}

\begin{frame}{Exemplo Clássico: Soma dos n primeiros naturais}
    
\textbf{Soma dos \( n \) primeiros números naturais:}
\[
\displaystyle \sum_{i=1}^{n} i = \frac{n(n+1)}{2}
\]
\end{frame}

\begin{frame}{Exemplo Clássico: Soma dos n primeiros naturais}

Exemplo: $\displaystyle \sum_{i=1}^{100} i = 1+2+3+...+99+100$
    
Observamos que: \hspace{1cm} $1 + 100 = 101$

\hspace{4cm}  $2 + 99 = 101$
    
\hspace{4cm} $3 + 98 = 101$

\vspace{0.2cm}

Encontramos $50$ resultados iguais a $101$, ou seja, a soma poderia ser encontrada fazendo $50 \cdot 101 = 5050.$

\vspace{0.2cm}

Podemos reescrever esse resultado na forma $\displaystyle \frac{100}{2} .(100+1).$

\vspace{0.2cm}

Então podemos conjecturar que $\displaystyle \sum_{i=1}^{n} i = \frac{n(n+1)}{2}.$

\vspace{0.2cm}

Para demonstração formal, utilizamos o Princípio da Indução Finita.

\end{frame}

\begin{frame}{Atividade}

\begin{itemize}
\item  Calcule o valor de $\displaystyle \frac{1+2+3+4+\cdots + 11}{3+4+5+\cdots + 12}$
\end{itemize}
\end{frame}

\begin{frame}{Atividade}

\begin{itemize}
		\item Prove que:
		\[
		\sum_{k=1}^{n} (2k-1) = n^2
		\]
\end{itemize}
    
\end{frame}


\begin{frame}{Exemplo Clássico: Soma dos n primeiros quadrados}
    \textbf{Soma dos quadrados dos \( n \) primeiros números naturais:}
    \[
    \displaystyle \sum_{i=1}^{n} i^2 = \frac{n(n+1)(2n+1)}{6}
    \]
\end{frame}

\begin{frame}{Exemplo Clássico: Soma dos n primeiros quadrados}
Sabemos que 
	
	$$
	(n+1)^3=n^3+3n^2+3n+1
	$$
	
	Então 
	
	para $n=1$, temos
	
	$$(1+1)^3 = 2^3 = 1^3+3.1^2+3.1+1$$
	
	para $n=2$, temos
	
	$$(2+1)^3 = 3^3 = 2^3+3.2^2+3.2+1$$
	
	para $n=3$, temos
	
	$$(3+1)^3 = 4^3 = 3^3+3.3^2+3.3+1$$

\end{frame}
    
\begin{frame}{Exemplo Clássico: Soma dos n primeiros quadrados}
	
	Vamos somar os termos seguintes e simplificar
	
	$$2^3 = 1^3+3.1^2+3.1+1$$
	$$3^3 = 2^3+3.2^2+3.2+1$$
	$$4^3 = 3^3+3.3^2+3.3+1$$
	$$5^3 = 4^3+3.4^2+3.4+1$$
	$$\vdots$$
	$$(n+1)^3=n^3+3n^2+3n+1$$
	
	ficamos com
	
	$$(n+1)^3 = 1^3 + 3.1^2+3.1+1 + 3.2^2+3.2+1 + 3.3^2+3.3+1 + ... +
	3n^2+3n+1$$

    \end{frame}

\begin{frame}{Exemplo Clássico: Soma dos n primeiros quadrados}
        
	ou seja
	$$(n+1)^3 = 1 + 3.(1^2 + 2^2 + 3^2 + 4^2 + ...+ n^2) + 3( 1+2+3+4+...+n) + n $$
	usando a fórmula para a soma dos n primeiros inteiros
	
	$$(n+1)^3 = 3.(1^2 + 2^2 + 3^2 + 4^2 + ...+ n^2) + 3\frac{n(n+1)}{2} + n + 1 $$
	multiplicando toda a equação por 2
	$$2(n+1)^3 = 2.3.(1^2 + 2^2 + 3^2 + 4^2 + ...+ n^2) + 3n(n+1) + 2(n+1) $$
	isolando a soma dos quadrados
	$$2(n+1)^3 - 3n(n+1) - 2(n+1) = 6(1^2 + 2^2 + 3^2 + 4^2 + ...+ n^2)$$
	colocando $(n+1)$ em evidência
	$$6(1^2 + 2^2 + 3^2 + 4^2 + ...+ n^2) = (n+1)[2(n+1)^2 - 3n - 2]$$
	
    \end{frame}

\begin{frame}{Exemplo Clássico: Soma dos n primeiros quadrados}
    
    Temos
	$$6(1^2 + 2^2 + 3^2 + 4^2 + ...+ n^2) = (n+1)[2(n^2+2n+1) - 3n - 2]$$
	
	$$6(1^2 + 2^2 + 3^2 + 4^2 + ...+ n^2) = (n+1)[2n^2+4n+2 - 3n - 2]$$
	
	$$6(1^2 + 2^2 + 3^2 + 4^2 + ...+ n^2) = (n+1)[2n^2+n]$$
	
	$$6(1^2 + 2^2 + 3^2 + 4^2 + ...+ n^2) = (n+1)n(2n+1)$$
	
	$$(1^2 + 2^2 + 3^2 + 4^2 + ...+ n^2) = \frac{n(n+1)(2n+1)}{6}$$
	Então podemos escrever
		\[
	\sum_{i=1}^{n} i^2 = \frac{n(n+1)(2n+1)}{6}
	\]
	
    
\end{frame}


\begin{frame}{Atividade}

\begin{itemize}
\item Calcule o valor de $\displaystyle (1+2+3+4+\cdots 10)^2$

\item Calcule $\displaystyle 1^2+2^2+3^2+4^2+\cdots + 10^2$ 
\end{itemize}
    
\end{frame}

\begin{frame}{Atividades}

Utilize as fórmulas de somatórios para calcular:

\begin{itemize}
		\item \( \displaystyle  \sum_{k=1}^{20} k \) 
		\item \( \displaystyle  \sum_{k=1}^{8} k^2 \) 
		\item \( \displaystyle  \sum_{k=1}^{10} k^3 \) (Dica: ultilize a fórmula para $(n+1)^4$ e repita o raciocínio feito para a soma dos quadrados)
\end{itemize}

\end{frame}
        


\begin{frame}{Aplicações}
    
    
    Essa notação é amplamente usada em diversas áreas da matemática e computação, como:
    \begin{itemize}
        \item Análise de algoritmos
        \item Estatística
        \item Inteligência artificial
    \end{itemize}
\end{frame}

\end{document}
